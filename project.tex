%%% Relativistic quantum fields notes.

\documentclass{momento}

\usepackage{danielphysics}
\usepackage{tensor}

\title{Dynamics, Electrodynamics, and Relativity}
\author{Daniel Williams}

\usetikzlibrary{arrows,matrix,positioning}
\providecommand{\Lag}{\mathcal{L}} %The Lagrangian
\providecommand{\Op}[1]{\mathrm{\hat{#1}}}  %A Quantum mechanical operator

\begin{document}
\frontmatter
{
\thispagestyle{empty}
\begin{tikzpicture}[remember picture,overlay]
  \fill[fill=accent-orange] (current page.south west) rectangle (current page.north east);
  %\fill[fill=white, yshift=-10cm]  (current page.north east) rectangle (current page.north west);
  \def\nbrcircles {377}
  \def\outerradius {30mm}
  \def\deviation {.9}
  \def\fudge {.62}

  \newcounter{cumulArea}
  \setcounter{cumulArea}{0}

  \pgfmathsetmacro {\goldenRatio} {(1+sqrt(5))}
  \pgfmathsetmacro {\meanArea} {pow(\outerradius * 10 / \nbrcircles, 2) * pi}
  \pgfmathsetmacro {\minArea} {\meanArea * (1 - \deviation)}
  \pgfmathsetmacro {\midArea} {\meanArea * (1 + \deviation) - \minArea}

  \foreach \b in {0,...,\nbrcircles}{
    % mod() must be used in order to calculate the right angle.
    % otherwise, when \b is greater than 28 the angle is greater
    % than 16384 and an error is raised ('Dimension too large').
    % -- thx Tonio for this one.
    \pgfmathsetmacro{\angle}{mod(\goldenRatio * \b, 2) * 180}

    \pgfmathsetmacro{\sratio}{\b / \nbrcircles}
    \pgfmathsetmacro{\smArea}{\minArea + \sratio * \midArea}
    \pgfmathsetmacro{\smRadius}{sqrt(\smArea / pi) / 2 * \fudge}
    \addtocounter{cumulArea}{\smArea};

    \pgfmathparse{sqrt(\value{cumulArea} / pi) / 2}
    \fill[opacity=0.3] (\angle:\pgfmathresult) circle [radius=\smRadius] ;
}  

  \node at (current page.center) [text width=16cm, yshift=10cm] 
    {\color{white}\fontsize{72pt}{106pt}\center \selectfont\sffamily Dynamics, Electrodynamics, and Relativity};

  \node at (current page.south) [text width= \textwidth, yshift=5cm] 
    {\color{white}\raggedleft{\fontsize{32pt}{120pt}\selectfont \sffamily Daniel Williams}};

\end{tikzpicture}
}
\newpage
\maketitle

\tableofcontents

%\part{Dynamics}
\label{par:introduction}

Classical mechanics is a subject which is hundreds of years old, and
despite its age continues to be an active research area.

Methods and structures in classical mechanics underly more modern
theories. When we think of classical mechanics we think of Newton's
Laws:
\begin{enumerate}
\item Every body persists in its state of being at rest or of moving uniformly straight forward, except insofar as it is compelled to change its state by force impressed.
\item The alteration of motion is ever proportional to the motive force impress'd; and is made in the direction of the right line in which that force is impress'd.
\item To every action there is always opposed an equal reaction: or the mutual actions of two bodies upon each other are always equal, and directed to contrary parts.
\end{enumerate}
These are easy to apply to simple systems, i.e. one or two particles
with one or two forces. It's difficult to apply these laws in
complicated systems, for example many-particle systems, and continuous
systems. The aim of this course is two-fold; first examining
non-familiar examples in Newtonian mechanics, e.g. non-inertial
frames; and develop an alternative formulation of classical mechanics,
the Lagrangian formalism, which is easier to apply to complex
systems.In particular new structures will become clear in this new
approach which do not appear easily in the Newtonian approach.

%%% Local Variables: 
%%% mode: latex
%%% TeX-master: "../project"
%%% End: 

\mainmatter
\chapter{Rotating Reference Frames}
\label{cha:rotat-refer-fram}

\section{Inertial Reference Frames}
\label{sec:inert-refer-fram}

An inertial frame is a coordinate system which is moving with a
constant velocity. The existence of inertial frames amounts to
Newton's first law.

Consider a frame $S(x,y,z)$ and another frame $S^\prime(x^{\prime},
y^{\prime}, z^\prime)$ with velocity $v$ with respect to $S$ along the
$+x$-direction.
The two frames are related by the equations
\begin{align}
  x^{\prime} &= x - vt \\
  y^{\prime} &= y \\
  z^\prime &= z \\
  t^{\prime} &= t
\end{align}
These are an example of a Galilean transformation. We could also have
used rotations.

Now, consider a particle with mass $m$. Observers in $S$ and
$S^{\prime}$ see

\begin{table}[H] \centering
\begin{tabular}{lcc} 
             & $S$                  & $s^{\prime}$               \\ \hline
Position     & $\vec{x}$            & $\vec{x}'$         \\ 
Velocity     & $\dv{\vec{x}}{t}$    & $\dv{\vec{x}'}{t}$ \\
Acceleration & $\dv[2]{\vec{x}}{t}$ & $\dv[2]{\vec{x}'}{t}$
\end{tabular}
\end{table}

But, from above, 

\begin{equation}%[Addition of Velocities]
 \dv{\vec{x}}{t} = 
 \begin{pmatrix}
   \dv{x}{t} \\ \dv{y}{t} \\ \dv{z}{t}
 \end{pmatrix}
=
\begin{pmatrix}
  \dv{x^{\prime}}{t} + v \\
 \dv{y^{\prime}}{t} \\
 \dv{z^{\prime}}{t}
\end{pmatrix}
\end{equation}

Thus, for velocities we interpret this to mean that the velocity of a
particle in another frame is equal to its velocity in its own frame
added to the relative velocity of its frame to the observer's.

Acceleration, on the other hand, is the same in both frames, and is a
frame-independent quantity. The general expression of this idea this
implies that the laws of physics are the same in all inertial frames;
the principle of Galilean relativity.

\section{Rotating Frames}
\label{sec:rotating-frames}

Consider a frame $S_1$ which is rotating with angular velocity,
$\omega$, about its $z$-axis with respect to an inertial frame, $S_0$,
such that the axes coincide when $t=0$. As such there will be an angle
between the sympathetic axes of $\theta = \omega t$ at any given time.

Now consider a vector $\vec{A}_1$ in $S_1$ which becomes 
\newcommand{\mat}[1]{\mathsf{#1}}
\[A_0 = \mat{R}(\omega t) \vec{A}_1 \] i.e. $\vec{A}$ as seen in $S_0$
rotates about the $z$-axis, corresponding to the rotation of the $S_1$
axes. The explicit form of $\mat{R}(\theta)$ is then
\[ \mat{R} = 
\begin{bmatrix}
  \cos(\omega t) & - \sin(\omega t) & 0 \\
\sin(\omega t)   & \cos(\omega t)   & 0 \\
0              & 0              & 1
\end{bmatrix}
\]

In general $A_1$ may vary with $t$, i.e. $A_1$ is not fixed in
$S_1$. Its rate of change in $S_1$ is not simply given by $A_1$. This
seems confusing at first but due to the fact that both the vector and
the coordinate axes in $S_1$ are changing with time. For example, a
fixed vector in $S_1$ has $\dot{A}_1=0$ but $\dot{A}_0 \neq 0$.

The rate of change of the axes introduces an additional term to
velocities in $S_0$; letting
\begin{align*} 
\dd{\vec{A}_0} &= \underbracket{\qty[ \dd{\mat{R}}(\omega t) ] \vec{A}_1}_{\text{From axis rotation.}} + \mat{R}(\omega t) \dd{\vec{A}_1} \\
&= \vec{\omega} \cp \vec{A}_0 \dd{t} + \mat{R}(\omega t) \dd{A_1} \\
&= \vec{\omega} \cp \qty[ \mat{R}(\omega t) \vec{A}_1] \dd{t} + \mat{R}(\omega t) \dd{A_1}
\end{align*}
Since
\[ \qty| \dd{\mat{R}} \vec{A}_1| = \omega \dd{t} \qty|
\vec{A}_0(\omega t) | = \qty| \omega \cp \vec{A}_0 | \dd{t} \] Where
the second part follows from the change in $\vec{A}_0$ being
perpendicular to $\vec{A}_0$ for the infinitessimal interval $\dd{t}$.
The quantity $\vec{\omega} \cp \vec{A}_0$ points in the direction of
$\dd{R} \vec{A}_1$, so given that the rotation is about the same axis
as the finite rotation $\mat{R}(\omega t)$,
\[ \omega \times \qty( \mat{R} \vec{A}_1) = \mat{R} \qty[ \vec{\omega}
\cp \vec{A}_1] \] This implies that the order of rotations is
irrelevant, so
\begin{align}
  \dd{\mat{R}} \vec{A}_1 &= \mat{R}(\omega t) \qty[ \vec{\omega} \cp \vec{A}_1] \dd{t} \nonumber \\
\implies \dd{\vec{A}_0} &= \mat{R}(\omega t) \qty[ \dd{\vec{A}_1} + \vec{\omega} \cp \vec{A}_1] \dd{t} \nonumber \\
\label{eq:2}
\dot{\vec{A}}_0 &= \mat{R}( \omega t ) \qty[ \dot{\vec{A}}_1 + \vec{\omega} \cp \vec{A}_1 ]
\end{align}

\section{The Coriolis and Centrifugal Forces}
\label{sec:cori-centr-forc}

Let a particle have positions $\vec{x}_0$ and $\vec{x}_1$ in the
frames $S_0$ and $S_1$ respectively, with $S_1$ rotating relative to
$S_0$ according to the transformation $\mat{R}(\omega t)$, so
\begin{align*}
  \vec{x}_0 &= \mat{R}(\omega t) \vec{x}_1 \\
&= \mat{R}(\omega t) \qty[ \dot{\vec{x}}_1 + \vec{\omega} \cp \vec{x}_1 ] &\omit\hfill \text{ (by eq. \eqref{eq:2})} \\
\implies \vec{v}_0 &= \mat{R}(\omega t) \qty[ \vec{v}_1 + \vec{\omega} \cp \vec{x}_1 ]
\end{align*}
Letting $\vec{A}_0 = \vec{v}_0$, and $\vec{A}_1 = \vec{v}_1 + \vec{\omega} \cp \vec{x}_1$, then
\begin{align*}
  \dot{\vec{v}}_0 &= \mat{R}(\omega t) \qty[ \dot{\vec{v}}_1 + \vec{\omega} \cp \dot{\vec{x}} + \vec{\omega} \cp \qty( \vec{v}_1 + \vec{\omega} \cp \vec{x}_1 ) ] \\ 
&= \mat{R}(\omega t) \qty[ \dot{\vec{v}}_1 + 2 \vec{\omega} \cp \vec{v}_1 + \vec{\omega} \cp ( \vec{\omega} \cp \vec{x}_1 )]
\end{align*}
Since $\vec{a}_i = \dot{\vec{v}}_i$ is the acceleration observed in
the frame $S_i$, then
\[ \vec{a}_1 = \mat{R}(-\omega t) \vec{a}_0 - 2 \vec{\omega} \cp
\vec{v}_1 - \vec{\omega} \cp ( \vec{\omega} \cp \vec{x}_1 ) \] So the
acceleration is not the same in each frame, and since in the
individual frames Newton's laws are valid, we see two additional
forces,
\begin{subequations}
\begin{align}
  \label{eq:3}
  \vec{f}~{cor} &= - 2 m \vec{\omega} \cp \vec{v}_1 \\
\vec{f}~{cen} &= - m \vec{\omega} \cp \qty( \vec{\omega} \cp \vec{x}_1 )
\end{align}
\end{subequations}
Respectively the Coriolis and Centrifugal forces.

\section{Motion at the Earth's Surface}
\label{sec:motion-at-earths}

The coordinate system on the Earth's surface is defined using two
coordinates, latitude, $\lambda$, and longitude, $\beta$. We also have
a rotation vector, $\vec{\Omega}$. A point close to the surface is
specified by three coordinates: altitude, latitude, and longitude, and
its velocity has the vector 
\[ \vec{v} = \qty( v~{long}, v~{lat}, v~{alt}) \]
an object near the Earth's surface will experience a Coriolis force,
\[ \vec{f}~{cor} = - 2m \vec{\Omega} \cp \vec{v} \]
and we have
\[ \vec{\Omega} = \qty( 0, \Omega \cos(\lambda), \Omega \sin(\lambda) ) \]
Thus
\begin{align} 
\vec{f}~{cor} &= -2m  \qty( 0, \Omega \cos(\lambda), \Omega \sin(\lambda) ) \cp \qty( v~{long}, v~{lat}, v~{alt}) \nonumber\\
&= - 2 m \Omega \begin{pmatrix} v~{alt} \cos(\lambda) - v~{lat} \sin(\lambda)\\ v~{long} \sin(\lambda) \\ - v~{long} \cos(\lambda)
\end{pmatrix} \nonumber \\
& \text{ if } v~{alt} = 0 \nonumber \\
&= -2m \Omega
\begin{pmatrix}
  - v~{lat} \sin(\lambda) \\ v~{long} \sin(\lambda) \\ 0 
\end{pmatrix} \nonumber \\
&= - 2m \qty(0 , 0 , \Omega \sin(\lambda) ) \cp \qty( v~{long}, v~{lat}, 0) \nonumber \\
\therefore \vec{f}~{cor} &= - 2 m \vec{\Omega}~{alt} \cp \vec{v}
\end{align}

\section{The Foucalt Pendulum}
\label{sec:foucalt-pendulum}

The plane of a pendulum's motion will rotate over time in a rotating
reference frame. Let $S_1$ be the rotating reference frame of the
Earth, and $S_2$ be a frame rotating with angular velocity $\omega_2
\propto \Omega~{alt}$. The accelerations in the two frames satisfy 
\begin{equation}
  \label{eq:1}
  \vec{a}_1 = \mat{R} \qty[ \vec{a}_2 + 2 \vec{\omega}_2 \cp \vec{v}_2 + \vec{\omega}_2 \cp ( \vec{\omega}_2 \cp \vec{x}_2 )]
\end{equation}
where $\vec{x}_i$, $\vec{v}_i$, and $\vec{a}_i$ are respectively the
position, velocity, and accleration in the $i$th frame. Assuming both
$\vec{\Omega}$ and $\vec{\omega}_2$ are small, then the centrifugal
term can be neglected, and
\[ \vec{a}_1 = \mat{R} \qty[ \vec{a}_2 + 2 \vec{\omega}_2 \cp \vec{v}_2 ] \]
we know
\[ \vec{a}_1 = \vec{g} - 2 \vec{\Omega}~{alt} \cp \vec{v}_2 \] For
$\vec{g}$ the acceleration due to gravity, so, given that the pendulum
undergoes horizontal motion, and we can neglect its vertical motion,
ignoring quadratic terms,
\[ \vec{v}_1 \approx \mat{R} \vec{v}_2 \]
and so
\[ \vec{g} - 2 \mat{R} \qty[ \vec{\Omega}~{alt} \cp \vec{v}_2 ] =
\mat{R} \qty[ \vec{a}_2 + 2 \vec{\omega}_2 \cp \vec{v}_2 ] \]
thus, multiplying both sides by $\mat{R}^{-1}$,
\begin{equation}
  \label{eq:4}
  \vec{a}_2 = \mat{R}^{-1} \vec{g}
\end{equation}
and so the acceleration of the bob is simply the acceleration due to
gravity transformed into a different frame, and in $S_2$ it appears as
if only gravity acts, so this represents a frame which is rotating to
counteract the rotation of the pendulum's plane, thus demonstrating
that the plane is rotating.
%%% Local Variables: 
%%% mode: latex
%%% TeX-master: "../project"
%%% End: 


\chapter{Motion of a System of Particles}
\label{cha:moti-syst-refer}
Consider a set of particles which each has a mass $m_i$, and position
vector $\vec{r}_i$, then each particle can experience two forces:
\begin{itemize}
\item $\vec{f}~{int}$ --- internal forces which act between particles in a system,
\item $\vec{f}~{ext}$ --- external forces which act on the system from
  outside, e.g. a magnetic flux.
\end{itemize}
The overall motion of a system is disjoint from its internal motions,
thanks to Newton's third law, since $F_{ij} = - F_{ji}$, where
$F_{ab}$ is the force exerted on particle $a$ by particle $b$, and the
force $F_{ij} \propto (\vec{r}_i - \vec{r}_j)$, i.e. acts upon the
shortest path between the particles.

\section{The Centre of Mass}
\label{sec:centre-mass}

The total mass of a system of particles is
\begin{equation}
  \label{eq:5}
  M = \sum_{i=1}^N m_i
\end{equation}
from which the total momentum can be found as 
\begin{equation}
  \label{eq:6}
  \vec{P} = \sum_{i=1}^N \vec{p}_i = \sum_{i=1}^N \dot{\vec{r}}_i m_i
\end{equation}
which, defining the centre of mass, 
\[ \vec{R} = \frac{\sum_i m_i \vec{r}_i}{\sum_i m_i}\]
leads to
\begin{equation}
  \label{eq:7}
  \vec{P} = M \dv{t} \vec{R}
\end{equation}
The force on the $i$th particle is then
\begin{equation}
  \label{eq:8}
  F_i = F_i^{\rm (ext)} + \sum_{j \neq i} \vec{F}_{ji} = \dv{t}\vec{P} 
\end{equation}
so
\[ \dv{t}\vec{P} = \sum_i \qty[ \vec{F}_i^{(\rm ext)} + \sum_{j \neq i} \vec{F}_{ji}] \]
We have
\begin{align*} \sum_i \sum_{j \neq i} \vec{F}_{ji} &= \sum_i \sum_{j < i} \vec{F}_{ji} + \sum_i \sum_{j > i} \vec{F}_{ji} \\ &= \sum_i \sum_{j<i} (\vec{F}_{ji} + F_{ij}) = 0
\end{align*}
Let $\vec{F}^{(\rm ext)} = \sum_i \vec{F}_i^{\rm (ext)}$, then
\begin{equation}
  \label{eq:9}
  F^{\rm (ext)} = \dv{t} P = M \dv[2]{t} \vec{R}
\end{equation}
Thus, the overall motion of a system relies only on the external force
acting upon it.

\section{Angular Momentum}
\label{sec:angular-momentum}

We have 
\[ m_i \dv[2]{t} r_i = \vec{F}_i^{\rm (ext)} + \sum_{j \neq i} \vec{F}_{ji} \]
and the total angular momentum of a system as
\begin{equation}
  \label{eq:10}
  \vec{L} = \sum^N_{i=1} \vec{r}_i \cp \vec{p}_i = \sum_{i=1}^N m_i \vec{r}_i \cp \dot{\vec{r}}_i
\end{equation}
Then
\begin{align}
  \dot{\vec{L}}          & = \dv{t} \sum_{i=1}^N m_i \vec{r}_i \cp \dot{\vec{r}} \nonumber                                                        \\
                         & = \sum_{i=1}^N m_i \dv{t} (\vec{r}_i \cp \dot{\vec{r}} )\nonumber                                                      \\
                         & = \sum_{i=1}^N m_i \vec{r}_i \cp \ddot{\vec{r}}_i \nonumber                                                            \\
                         & = \sum_{i=1}^N \vec{r}_i \cp \vec{F}_i^{\rm (ext)} + \sum_{i=1}^N \sum_{j \neq i} \vec{r}_i \cp \vec{F}_{ji} \nonumber \\
                         & = \vec{G}^{\rm (ext)} + \sum_{i=1}^N \sum_{j>i} (\vec{r} \cp \vec{F}_{ji} + \vec{r}_j \cp \vec{F}_{ij} ) \nonumber     \\
                         & = \vec{G}^{\rm (ext)} + \sum_{i=1}^N \sum_{j>i} (\vec{r}_i - \vec{r}_j) \cp \vec{F}_{ji} \nonumber                      \\
\therefore \dot{\vec{L}} & = \vec{G}^{\rm (ext)}
\end{align}
Thus angular momentum is conserved unless there is an applied torque.

\section{Separation of Kinetic Energy}
\label{sec:separ-kinet-energy}

Let the position of a particle be described relative to the centre of mass, i.e.
\[ \vec{r}_i = \vec{R} + \vec{r}'_i \]
Then
\begin{align*}
  \sum_{i=1}^N m_i \vec{r}'_i &= \sum^N_{i=1} m_i \vec{r}_i - \sum_{i=1}^N m_i R \\
&= M \qty[ \frac{\sum m_i \vec{r}_i}{\sum m_i } - \vec{R}] = 0
\end{align*}
the kinetic energy $T$ is then
\begin{align}
  T &= \half \sum m_i \dv{\vec{r}_i}{t}^2 \nonumber\\
&= \half \sum_{i=1}^N m_i \qty[ \dot{R}^2 + 2 \dot{\vec{r}}'_i \vdot \dot{\vec{R}} + (\dot{\vec{r}}'_i)^2 ] \nonumber\\
&= \half \sum m_i \dot{\vec{R}}^2 + \half \sum m_i \dot{\vec{r}'_i}^2 + \sum m_i \dot{\vec{r}_i'} \nonumber\\
&= \half \sum m_i \dot{\vec{R}}^2 + \half \sum m_i (\dot{\vec{r}}'_i)^2
\end{align}
Thus the kinetic energy is the sum of the internal energies and the
kinetic energy of a single particle with the mass of the whole system.

\section{Separation of Angular Momentum}
\label{sec:separ-angul-moment}

The total angular momentum of a system is
\begin{align}
  \vec{L} &= \sum \vec{r}_i \cp \vec{p}_i \nonumber \\ &= \sum m \vec{r}_i \cp \dot{\vec{r}}_i \nonumber\\
&= \sum m_i (\vec{R} + \vec{r}_i') \cp (\dot{\vec{R}} + \dot{\vec{r'}}_i ) \nonumber\\
&= M \vec{R} \cp \dot{\vec{R}} + \qty[ \sum m_i \vec{r}' ] \cp \dot{\vec{R}} \nonumber\\ & \quad+ \vec{R} \cp \qty[ \sum m_i \dot{\vec{r}}'_i ] + \sum m_i \vec{r}'_i \cp \dot{\vec{r}}'_i \nonumber\\
&= M \vec{R} \cp \dot{\vec{R}} + \vec{L}~{int}
\end{align}
Where $\vec{L}~{int} = \sum m_i \vec{r}'_i \cp \dot{\vec{r}}'_i$, so
\begin{align}
  \dot{\vec{L}} &= \sum r_i \cp \vec{F}_i^{\rm (ext)} \nonumber\\
 &= \underbracket{\vec{R} \cp \sum \vec{F}_i^{\rm (ext)}}_{\text{torque on system}} + \sum \underbracket{\vec{r}'_i \cp \vec{F}_i^{\rm (ext)}}_{\text{torque on each particle}}
\end{align}

%%% Local Variables: 
%%% mode: latex
%%% TeX-master: "../project"
%%% End: 


\chapter{The Lagrangian Formalism}
\label{cha:lagrangian-formalism}

\section{Constraints}
\label{sec:constraints}

The motion of a particle may be restricted in an arbitrary way via
contraints, for example, a pendulum attached to a string of length $L$
has a contraint 
\[ x^2 + y^2 = L^2 \] In general a contraint can be characterised as a
force, in the case of the pendulum this is tension.

A \textbf{holonomic} constraint has the form
\[ g(\vec{x}_1, \vec{x}_2, \dots, \vec{x}_n, t) = 0 \] If the
constraint has time-dependence it is \textbf{rheonomous}, otherwise it
is \textbf{scleronomous}. Not every contraint is holonomic; the
constraint that a particle may not enter a sphere is an inequality,
and thus non-holonomic, for example.

\section{Generalised Coordinates}
\label{sec:gener-coord}

Generalised coordinates are chosen to simplify the description of a
system, and are linearly independent. These are a set of independent
quantities $\set{q_i}$, such that
\begin{equation}
  \label{eq:11}
  \vec{x}_i = \vec{x}_i(\set{q_i}, t) 
\end{equation}

The number of degrees of freedom, $f$, for a system is equal to the product
of the system's dimensionality, $D$, and its number of particles, $N$,
less the number of constraints, i.e.
\[ f = DN - k \] and this results in a system requiring $f$
generalised coordinates to fully characterise it.

\section{Virtual Work $\star$}
\label{sec:virtual-work}

A virtual displacement of a system is a change in a system's
configuration as a result of any arbitrary infinitessimal change of
its coordinates, $\delta \vec{r}_i$.

The virtual work of a force $\vec{F}_i$ is $\vec{F}_i \vdot \delta
\vec{r}_i$. If the system is in equilibrium then $\vec{F}_i = 0$, so
the virtual work also vanishes. For the whole system
\[ \sum \vec{F}_i \vdot \delta \vec{r}_i = 0 \] We can express a force
in the form $\vec{F}_i = \vec{F}_i^{\rm (a)} +\vec{f}_i$, the sum of
an applied force and a constraint, so
\begin{equation}
  \label{eq:12}
  \sum \vec{F}_i^{\rm (a)} \vdot \delta \vec{r}_i + \sum_i \vec{f}_i \vdot \delta \vec{r}_i =0
\end{equation}
If the constraint forces produce no net virtual work (excluding e.g. sliding friction),
\begin{equation}
  \label{eq:13}
  \sum \vec{F}_i^{\rm (a)} \vdot \delta \vec{r}_i = 0
\end{equation}
Which is the principle of virtual work.

Thanks to the constraints $\delta \vec{r}_i$ are not independent, so a
form for the general motion of the system in general coordinates is
required.

\section{D'Alembert's Principle}
\label{sec:dalemberts-principle}

Take the equation of motion, 
\[ \vec{F}_i = \dot{\vec{p}}_i \]
which can be expressed 
\[ \vec{F}_i - \dot{\vec{p}}_i = 0 \]
Then
\begin{subequations}
\begin{align}
  \label{eq:14}
  \sum (\vec{F}_i - \dot{\vec{p}}_i) \vdot \delta \vec{r}_i &=0 \\
\sum_i (\vec{F}_i^{\rm (a)} - \dot{\vec{p}}_i ) \vdot \delta \vec{r}_i + \sum \vec{f}_i \vdot \delta \vec{r}_i &= 0 \\
\sum_i (\vec{F}_i^{\rm (a)} - \dot{\vec{p}}_i ) \vdot \delta \vec{r}_i &= 0
\end{align}
\end{subequations}
which is \emph{D'Alembert's Principle}. To convert this to generalised
coordinates we use a transformation
\begin{equation}
  \label{eq:15}
  \delta \vec{r}_i = \sum_j \pdv{\vec{r}_i}{q_j} \delta q_j
\end{equation}
using the summation convention, and defining $\Lambda^j_i = \pdv{\vec{r}_i}{q_j}$,
\[ \delta \vec{r}_i = \Lambda^j_i \delta q_j \]
Then the virtual work becomes
\begin{equation}
  \label{eq:16}
  \sum_i \vec{F}_i  \vdot \delta \vec{r}_i = \sum_{i,j} \vec{F}_i \vdot \Lambda^j_i \delta q_j = \sum_j Q_j \delta q_j
\end{equation}
with $Q_j$ the components of a virtual force,
\[ Q_j = \sum_i \vec{F}_i \Lambda^i_j \]

We also have the reversed effective force in equation \eqref{eq:14},
\begin{equation}
  \label{eq:17}
  \sum \dot{\vec{p}}_i \vdot \delta \vec{r}_i = \sum m_i \ddot{\vec{r}}_i \vdot \delta \vec{r}_i = \sum_{i,j} m_i \ddot{\vec{r}}_i \Lambda^j_i \delta q_j 
\end{equation}

Then
\begin{align*}
  \sum m_i \ddot{\vec{r}}_i \vdot \Lambda_i^j &= \sum_i \qty[ \dv{t} \qty(m_i \dot{\vec{r}} \vdot \Lambda_i^j ) - m_i \dot{\vec{r}} \vdot \dv{t} \qty( \Lambda^j_i )  ] \\ 
&= \sum_i \qty[ \dv{t} \qty( m_i \vec{v}_i \vdot \pdv{\vec{v}_i}{\dot{q}_j} ) - m_i \vec{v}_i \pdv{\vec{v}_i}{q_j}]
\end{align*}
Since 
\begin{align*}
  \dv{t} \pdv{\vec{r}_i}{q_j} &= \pdv{\vec{v}}{q_j} \\
\pdv{\vec{v}_i}{\dot{{q}}_j} &= \pdv{\vec{r}_i}{q_j}
\end{align*}
Then equation \eqref{eq:17} can be expanded to
\begin{align*} 
  \sum_j & \bigg\{
    \dv{t} \qty[ 
      \pdv{\dot{q}_j} \qty( \sum_i \half m_i v_i^2) 
    ]
    \\ & \quad- \pdv{q_j} \qty( \sum_i \half m_i v_i^2 )
    - Q_j
  \bigg\} \delta q_j
\end{align*}
Then
\begin{equation}
  \label{eq:18}
  \sum \qty{ \qty[
    \dv{t} \qty( \pdv{T}{\dot{q}_j} ) - \pdv{T}{q_j}
  ] - Q_j } \delta q_j = 0
\end{equation}
For $T$ the kinetic energy, and so,
\begin{equation}
  \label{eq:19}
  \dv{t} \qty( \pdv{T}{\dot{q}_j} ) - \pdv{T}{q_j} = Q_j 
\end{equation}
When the forces are produced by a potential, $\vec{F}_i = - \nabla_i
V$,
\[ Q_j = - \sum_i \nabla_i V \vdot \pdv{\vec{r}_i}{q_j} = - \pdv{V}{q_i}\]
so we now have
\begin{equation}
  \label{eq:20}
  \dv{t} \qty( \pdv{T}{\dot{q}_j} ) - \pdv{(T-V)}{q_i} = 0
\end{equation}
and defining a function $L = T - V$, the \emph{Lagrangian}, and noting
that $\pdv{V}{\dot{q}_j} = 0$, we get

\begin{equation}
  \label{eq:21}
  \dv{t} \qty( \pdv{L}{\dot{q}_j} ) - \pdv{L}{q_j} = 0
\end{equation}
which are \emph{Lagrange's equations}.

\section{Velocity-dependent Potentials $\star$}
\label{sec:veloc-depend-potent}

Suppose there is no potential $V$ to generate the generalised forces,
but they can instead be found from a function $U(q_j, \dot{q_j})$ by
\begin{equation}
  \label{eq:22}
  Q_j = - \pdv{U}{q_j} + \dv{t}\qty( \pdv{U}{\dot{q}_j} )
\end{equation}
The Lagrangian is now
\[ L = T-U \] and $U$ is a ``generalised potential''. Such a potential
is of importance in electromagnetism.

\subsection{The Electromagnetic Vector Potential}
\label{sec:electr-vect-potent}

Consider the force on a charge,
\begin{equation}
  \label{eq:23}
  \vec{F} = q \qty[ \vec{E} + \vec{v} \cp \vec{B}]
\end{equation}
for $\vec{E} = \vec{E}(x,y,z,t)$ and $\vec{B} = \vec{B}(x,y,z,t)$
being continuous functions of time and position. These can be derived
from a scalar potential and a vector potential, respectively
$\phi(t,x,y,z)$ and $\vec{A}(t,x,y,z)$:
\begin{subequations}
\begin{align}
\vec{E} & = - \nabla \phi - \pdv{\vec{A}}{t} \\ 
\vec{B} & = \nabla \cp \vec{A}  
\end{align}
\end{subequations}
The force can then be derived from the potential $U$,
\begin{equation}
  \label{eq:24}
  U = q \phi - q \vec{A} \vdot \vec{v}
\end{equation}
and the Lagrangian is then
\begin{equation}
  \label{eq:25}
  L = \half m v^2 - q \phi + q \vec{A} \vdot \vec{v}
\end{equation}
This can then be used to derive equation \eqref{eq:23}.

\section{Dissipation $\star$}
\label{sec:dissipation}

If not all of the forces in a system can be derived from a potential
then
\begin{equation}
  \label{eq:26}
  \dv{t} \qty( \pdv{L}{\dot{q}_j} ) - \pdv{L}{q_j} = Q_j
\end{equation}
This happens in the case of friction, where there is a force 
\[ F_{{\rm f}x} = - k_x v_x \]
Such a force can be considered by the dissipation function,
\begin{equation}
  \label{eq:27}
  \mathcal{F} = \half \sum_i \qty( k_x v_{ix}^2 + k_y v_{iy}^2 + k_z v_{iz}^2 )
\end{equation}
where 
\[ \vec{F}_{{\rm f}x} = - \pdv{\mathcal{F}}{v_x} \implies \vec{F}~{f} = - \nabla_v \mathcal{F} \]
The Lagrange equations then become
\begin{equation}
  \label{eq:28}
  \dv{t} \pdv{L}{\dot{q}_j} - \pdv{L}{q_j} + \pdv{\mathcal{F}}{\dot{q}_j} = 0 
\end{equation}

\section{Hamilton's Principle}
\label{sec:hamiltons-principle}

It is possible to derive the Lagrange equations for a system from an
integrated perspective of the motion, using Hamilton's Principle (the
principle of least action):
\begin{quotation}
  The motion of a system from a time $t_1$ to $t_2$ is such that the line integral 
  \[ I = \int_{t_1}^{t_2} L \dd{t} \] for $L = T - V$ has a stationary
  value for the actual path of the motion.
\end{quotation}

We define the action as the integral from Hamilton's Principle,
\begin{equation}
  \label{eq:29}
  S( \set{q_i}, \set{\dot{q}_i} ) = \int_{t_0}^{t_1} \dd{t} L(\set{q_i}, \set{\dot{q}_i}, t)
\end{equation}

To derive the Lagrange equations introduce a small perturbation to
$q_i$,
\[ q_i(t) \to q_i(t) + \delta q_i(t) \] The trajectory has fixed
end-points, so $\delta q_i(t_0) = \delta q_i(t_1) = 0$, so
\[  \dot{q}_i \to q_i + \delta \dot{q}_i, \quad \delta \dot{q}_i = \dv{t} \delta q_i \]

This perturbs the action,
\[ S \to S + \var{S} = \int_{t_0}^{t_1} \dd{t} L(q_i+\var{q_i}, \dot{q}_i + \var{\dot{q}_i}, t) \]
Using Taylor's theorem,
\begin{equation}
  \label{eq:30}
  S + \var{S} = \int_{t_0}^{t_1} \dd{t} \qty[ L + \sum_i \qty( \pdv{L}{q_i} \var{q_i} + \pdv{L}{\dot{q}_i} \var{\dot{q}_i} ) ] + \cdots
\end{equation}
Then
\begin{align*}
  \label{eq:31}
  \var{S} &= \sum_i\int_{t_0}^{t_1} \dd{t} \qty[  \pdv{L}{q_i} \var{q_i} + \pdv{L}{\dot{q}_i} \var{\dot{q}_i}  ] \\
&= \sum_i \qty{ \underbracket{\eval{ \pdv{L}{\dot{q}_i} \var{q_i} }_{t_0}^{t_1}}_{=0} + \underbracket{\int_{t_0}^{t_1} \dd{t} \qty[\pdv{L}{q_i} - \dv{t} \pdv{L}{\dot{q}_i} ] \delta q_i  }_{\text{Must be zero to extremise  action}} } \\
\therefore \dv{t} \pdv{L}{\dot{q}_i}&= \pdv{L}{q_i}
\end{align*}

The variational approach to finding the Lagrangian allows easy
extension to systems which are not normally the domain of dynamics,
for example the descriptions of electrical circuits.

\section{Canonical Momentum}
\label{sec:canonical-momentum}

Recall the Lagrangian for a particle moving in one dimension,
\[ L = \half m \dot{x}^2 - V(x) \]
from which
\[ \pdv{L}{x} = m \dot{x} \] which is the particle's momentum. Given a
general Lagrangian the quantity
\begin{equation}
  \label{eq:32}
  p_i = \pdv{L}{\dot{q}_i}
\end{equation}
is the generalised momentum, or the canonically conjugate momentum to
$q_i$.

\section{Symmetries and Conservation}
\label{sec:symm-cons-theor}

If the Lagrangian of a system doesn't explicity contain a coordinate
$q_i$, (but may contain $\dot{q}_j$) it is called cyclic, so
\begin{equation}
  \label{eq:33}
  \pdv{L}{q_i} = 0 \implies \dv{t}\pdv{L}{\dot{q}_i} = \dot{p}_i = 0
\end{equation}
Therefore, the momentum conjugate to a cyclic coordinate is conserved.

\section{The Energy Function}
\label{sec:energy-function}

Consider the time derivative of $L$,
\begin{align*}
  \label{eq:34}
  \dv{L}{t} &= \sum_i \pdv{L}{q_i} \dv{q_i}{t} + \sum_i \pdv{L}{\dot{q}_i} \dv{\dot{q}_i}{t} + \pdv{L}{t} \\
&= \sum_j \dv{t} \qty(\pdv{L}{\dot{q}_j}) \dot{q}_j + \sum_j \pdv{L}{\dot{q}_j} \dv{\dot{q}_j}{t} +\pdv{L}{t}
\end{align*}
Thus
\begin{align*}
\dv{t} \qty( \sum_j \dot{q}_j \pdv{L}{\dot{q}_j} - L ) + \pdv{L}{t} &= 0 \\
\dv{H}{t} &= - \pdv{L}{t}
\end{align*}
For 
\begin{equation}
  \label{eq:35}
  h = \sum_i \dot{q}_i \pdv{L}{\dot{q}_i} - L
\end{equation}
defined as the ``Energy function'', which is physically, if not mathematically, identical to the Hamiltonian.
Thus
\begin{equation}
  \label{eq:92}
  \dv{h}{t} = - \pdv{L}{t}
\end{equation}
If the Lagrangian is not explicitly a function of time then $h$ is
conserved. This is one of the first integrals of the motion, and is 
\emph{Jacobi's integral}.

Under some circumstances $h$ is the total energy of a system; recall
that the total kinetic energy of a system can be expressed
\begin{equation}
  \label{eq:93}
  T = T_0 + T_1 + T_2
\end{equation}
with $T_0$ a function of only the generalised coordinates, $T_1(q,
\dot{q})$ is linear in the generalised velocities, and $T_2(q,
\dot{q})$ is quadratic in $\dot{q}$. For a wide range of systems we
can similarly decompose the Lagrangian,
\begin{equation}
  \label{eq:94}
  L = L_0 + L_1 + L_2
\end{equation}
If a function $f$ is homogeneous and of degree $n$ in the variables
$x_i$, then
$
  \sum_i x_i \pdv*{f}{x_i} = nf
$
applied to the function $h$,
\begin{equation}
  \label{eq:96}
  h = 2 L_2 + L_1 - L = L_2 - L_0
\end{equation}
If the transformations to generalised coordinates are time independent $T = T_2$, and then if the potential doesn't depend on generalised velocity, $L_0 = -V$, so
\begin{equation}
  \label{eq:97}
  h = T + V = E
\end{equation}
%%% Local Variables: 
%%% mode: latex
%%% TeX-master: "../project"
%%% End: 


\chapter{Small Oscillations}
\label{cha:small-oscillations}

\section{The Double Pendulum}
\label{sec:double-pendulum}

Consider a double pendulum, consisting of two bobs, one hung below the
other. Each has length $a$, and bobs of mass $m$, so the potential
energy is
\begin{equation}
  \label{eq:36}
  V = -mga \cos(\theta) - mga \qty( \cos(\theta) + \cos(\phi) )
\end{equation}
and the kinetic energy is
\begin{equation}
  \label{eq:36}
  T = \half m a^2 \dot{\theta}^2 + \half ma^2 (\dot{\theta} + \dot{\phi})^2
\end{equation}
Then,
\begin{align*}
  L & = \half ma^2 \dot{\theta}^2 + \half ma^2 (\dot{\theta}^2 + \dot{\phi}^2 + 2 \dot{\theta} \dot{\phi}) \\
    &                          \qquad  + mga (2 \cos(\theta) + \cos(\phi) ) \\
&\approx \half m a^2 \dot{\theta}^2 + \half m a^2 (\dot{\theta}^2 + \dot{\phi}^2 + 2 \dot{\theta} \dot{\phi}) \\
& \qquad - mga \qty(\theta^2 + \half \phi^2 )
\end{align*}
The equations of motion from the Lagrange equations are
\begin{subequations}
  \begin{align}
    \label{eq:36}
2 \ddot{\theta} + \ddot{\phi} + \frac{2g}{a} \theta & =0 \\
\ddot{\theta} + \ddot{\phi} + \frac{g}{a} \phi &= 0
  \end{align}
\end{subequations}
Each of these equations has a form comparable to that of a harmonic
oscillator, $\ddot{x} + \omega^2 x = 0$; attempting a trial solution
\[
\begin{bmatrix}
  \theta \\ \phi
\end{bmatrix}
=
\begin{bmatrix}
  c_{\theta} e^{i \omega t} \\ c_{\phi} e^{i \omega t}
\end{bmatrix}
\]
For $c_{\theta}$, $c_{\phi}$ complex constants, then
\[ \ddot{\theta} = - \omega^2 \theta, \qquad \ddot{\phi} = - \omega^2 \phi \]
So
\begin{subequations}
  \begin{align}
\label{eq:39}
    \qty( - 2 \omega^2 + \frac{2g}{a} ) c_{\theta} - \omega^2 c_{\phi} &= 0 \\
\label{eq:38}
- \omega^2 c_{\theta} + \qty( -\omega^2 + \frac{g}{a} ) c_{\phi} &= 0
  \end{align}
Which can be expressed in matrix notation
\begin{equation}
  \label{eq:37}
  \begin{bmatrix}
    2 \frac{g}{a} - 2 \omega^2 & - \omega^2 \\ - \omega^2 & \frac{g}{a} - \omega^2
  \end{bmatrix}
  \begin{bmatrix}
    c_{\theta} \\ c_{\phi}
  \end{bmatrix} = 0
\end{equation}
\end{subequations}
This implies that the determinant of the matrix must be zero, so
\begin{equation}
  \label{eq:40}
    \begin{vmatrix}
    2 \frac{g}{a} - 2 \omega^2 & - \omega^2 \\ - \omega^2 & \frac{g}{a} - \omega^2
  \end{vmatrix} = 2 \qty( \frac{g}{a} - \omega^2 )^2 - \omega^4 = 0
\end{equation}
This has two solutions,
\begin{equation}
  \label{eq:41}
  \omega^2 = \frac{g}{a} ( 2 \pm \sqrt{2})
\end{equation}
which are the normal frequencies for the system, and the coordinates
$c_i$ are the normal modes. To find these we substitute the normal
frequencies into equation \eqref{eq:37},
\begin{equation}
  \label{eq:42}
  \frac{g}{a}
  \begin{bmatrix}
    2-4-2 \sqrt{2} & -2-\sqrt{2} \\ -2 -\sqrt{2} & 1-2-\sqrt{2}
  \end{bmatrix}
  \begin{bmatrix}
    c_{\theta} \\ c_{\phi} 
  \end{bmatrix} = 0 
\end{equation}
These turn out to give two copies of the same equation relating the
coefficients, so clearly only the relative relation of them is fixed,
\begin{equation}
  \label{eq:43}
  \frac{c_{\theta}}{c_{\phi}} = - \frac{2 + \sqrt{2}}{2(1+\sqrt{2})} = - \frac{(2+\sqrt{2})(1-\sqrt{2})}{2(1+\sqrt{2})(1-\sqrt{2})} = -\frac{1}{\sqrt{2}}
\end{equation}
Thus 
\begin{equation}
  \label{eq:44}
    \begin{bmatrix}
    c_{\theta} \\ c_{\phi} 
  \end{bmatrix} \propto
  \begin{bmatrix}
    -1 \\ \sqrt{2}
  \end{bmatrix}
\end{equation}
and using the negative solution
\begin{equation}
  \label{eq:45}
    \begin{bmatrix}
    c_{\theta} \\ c_{\phi} 
  \end{bmatrix} \propto
  \begin{bmatrix}
    1 \\ \sqrt{2}
  \end{bmatrix}
\end{equation}
Giving a general solution
\begin{equation}
  \label{eq:45}
  \begin{bmatrix} \theta \\ \phi \end{bmatrix}
= \alpha_1 \begin{bmatrix}  -1 \\ \sqrt{2}  \end{bmatrix} e^{i \omega_1 t} + \alpha_2 \begin{bmatrix}  1 \\ \sqrt{2}  \end{bmatrix} e^{i \omega_2 t}
\end{equation}
This can be rewritten in matrix form too,
\begin{equation}
  \label{eq:46}
    \begin{bmatrix} \theta \\ \phi \end{bmatrix} = 
    \begin{bmatrix} - 1 & 1 \\ \sqrt{2} & \sqrt{2} \end{bmatrix}
    \begin{bmatrix} \alpha_1 e^{i \omega_1 t} \\ \alpha_2 e^{i \omega_2 t}    \end{bmatrix}
\end{equation}
This can be inverted, giving
\begin{equation}
  \label{eq:47}
    \begin{bmatrix} \alpha_1 e^{i \omega_1 t} \\ \alpha_2 e^{i \omega_2 t}    \end{bmatrix} =
    \begin{bmatrix} - \half & \half \sqrt{2} \\ \half & \half \sqrt{2} \end{bmatrix}
    \begin{bmatrix} \theta \\ \phi \end{bmatrix} =
    \begin{bmatrix} \xi_1 \\ \xi_2 \end{bmatrix}
\end{equation}
For $\xi_i$ the \emph{normal coordinates} of the system, these cause
the Lagrange equations to completely decouple.

\section{General Theory of Small Oscillations}
\label{sec:general-theory-small}

Consider a system with time-independent constraints; this is in equilibrium if
\[ \pdv{V}{q_i} = 0 \]
Furthermore, a stable equilibrium has \[ \pdv[2]{V}{q_i}{q_j} > 0 \quad \forall i, j\]

Denoting the equilibrium value of each coordinate $q^{*}_i$, we can
introduce a small perturbation, $\eta_i$, such that
\begin{equation}
  \label{eq:48}
  q_i = q^{*}_i + \eta_i
\end{equation}
Assuming small displacements we can use Taylor's theorem to expand the
potential about $q_i = q_i^{*}$:
\begin{equation}
  \label{eq:49}
  V = V^{*} + \sum \eta_i \pdv{V}{q_i} + \sum_{i,j} \half \qty( \pdv[2]{V}{q_i}{q_j} ) \eta_i \eta_j
\end{equation}
to the second-order.
The potential thus has the form of the second derivative term,
\begin{equation}
  \label{eq:50}
  V = \sum_{i,j} \half V_{,ij} \eta_i \eta_j
\end{equation}
Using the comma notation for derivatives. The matrix $\mat{V}$ has
components $V_{,ij}$, and the set of displacements $\eta_i$ forms a
vector $\vec{\eta}$, so
\begin{equation}
  \label{eq:51}
  V = \half \trans{\eta} \mat{V} \eta
\end{equation}
Since $\mat{V}$ doesn't depend upon the coordinates, just the
equilibrium values, it is constant. The kinetic energy has the form
\begin{equation}
  \label{eq:52}
  T = \sum_{i,j} \half m_{ij} \dot{q}_i \dot{q}_j
\end{equation}
Expanding about the equilibrium we find
\[ \dot{q}_i = \dot{\eta}_i \] and
\[ m_{ij}(q_1, \dots, q_n) = m_{ij}(q_1^{*}, \dots, q_n^{{*}}) + \cdots \]
Then
\begin{equation}
  \label{eq:53}
  T = \sum_{i,j} \half T_{ij} \dot{\eta}_i \dot{\eta}_j
\end{equation}
having defined $T_{ij} = m_{ij}(q^*_1, \dots, q^{*}_n)$, which is a
constant matrix, so we have
\begin{equation}
  \label{eq:54}
  T = \half \trans{\dot{\eta}} \mat{T} \dot{\eta}
\end{equation}
and
\begin{equation}
  \label{eq:55}
  L = T-V = \sum_{i,j} \half \qty[ T_{ij} \dot{\eta}_i \dot{\eta}_j - V_{ij} \eta_i \eta_j] = \half \trans{\dot{\eta}} \mat{T} \dot{\eta} - \half \trans{\eta} \mat{V} \eta
\end{equation}
The Lagrange equations have the form
\[ \dv{t} \pdv{L}{\dot{\eta}_k} = \pdv{L}{\eta_k} \]
Taking this in bits,
\[ \pdv{L}{\dot{\eta}_k} = \half \sum_{i,j} T_{ij} \qty[ \pdv{\dot{\eta}_i}{\dot{\eta}_k} + \dot{\eta}_i \pdv{\dot{\eta}_j}{\dot{\eta}_k}] \]
The generalised coordinates are independent, so
\[ \pdv{\dot{\eta}_i}{\dot{\eta}_k} = \delta_{ik} \]
Thus
\begin{subequations}
\begin{equation}
  \label{eq:56}
  \pdv{L}{\dot{\eta}_k} = \half \sum_{i,j} T_{ij} \qty[ \delta_{ik} \dot{\eta}_j + \dot{\eta}_i \delta_{jk}] = \sum_j T_{kj} \dot{\eta}_j
\end{equation}
Similarly,
\begin{equation}
  \label{eq:57}
  \pdv{L}{\eta_k} = - \sum_j V_{,kj} \eta_j
\end{equation}
\end{subequations}
This gives the Lagrange equations in the form
\begin{equation}
  \label{eq:58}
  \sum_j (T_{ij} \ddot{\theta}_j + V_{,ij} \eta_j) = 0 \equiv \mat{T} \ddot{\vec{\eta}} + \mat{V} \vec{\eta}_j = \vec{0}
\end{equation}
As in the double pendulum case, we can find solutions of the form
\[ \eta = \vec{c} e^{i \omega t} \]
Then $\ddot{\eta} = - \omega^2 \eta$, and so
\begin{equation}
  \label{eq:59}
  \qty( \mat{V} - \omega^2 \mat{T} ) \eta = \vec{0}
\end{equation}
To satisfy the equation we need
\begin{equation}
  \label{eq:60}
  \abs{\mat{V}-\omega^2 \mat{T}} = 0
\end{equation}
which is a characteristic equation, and this can be approached as an
eigenvalue equation, and once the normal frequencies are found we
substitute them back in turn, and find the vectors $\vec{c}$ by
solving \[ \mat{V} \vec{c} = \omega^2 \mat{T} \vec{c} \] for each
$\omega^2$. THis is akin to finding the eigenvectors of a matrix, and
the vectors specify the \emph{normal modes} of the oscillation.

In general, from the fact that $\mat{V}$ and $\mat{T}$ are symmetric,
and that $\omega^2$ has real solutions, that $\vec{c}$ can be chosen
to be orthonormal. The general solution takes the form
\begin{equation}
  \label{eq:80}
  \begin{pmatrix}
    \eta_1 \\ \vdots \\ \eta_n
  \end{pmatrix}
 = \sum_j \alpha_j \vec{c}^{(j)} \exp( i \omega_j t) = \sum_j \alpha_j 
 \begin{pmatrix}
   c_1^{(j)} \\ \vdots c_n^{(j)}
 \end{pmatrix}
\exp(i \omega_j t)
\end{equation}
which can be expressed more compactly, using Einstein notation,
\begin{equation}
\label{eq:81}
 \eta_i = c_i^j \alpha_j \exp(i \omega_j t)
\end{equation}
for $c_i^j = c_i^{(j)}$. We can then define normal coordinates,
\[ \xi_j = \alpha_j \exp(i \omega_j t) \] which correspond to the
oscillation of the system at a single frequency, and can be found by
inverting equation (\ref{eq:80}), so
\begin{equation}
  \label{eq:82}
  \xi_j = (c_i^j)^{-1} \eta_i
\end{equation}
The Lagrangian then completely decouples into the sum of independent
harmonic oscillators,
\begin{equation}
  \label{eq:87}
  L = \sum_j C_j \qty[ \dot{\xi}_j^2 - \omega^2_j \xi_j^2]
\end{equation}
for $C_j$ a normalisation constant.


%%% Local Variables: 
%%% mode: latex
%%% TeX-master: "../project"
%%% End: 


\chapter{The Lagrangian Formulation for Fields $\star$}
\label{cha:lagr-form-fields}
\tikzset{
	%%% SPRING STYLING
 	set spring/.code={\pgfqkeys{/tikz/spring}{#1}},
 	set spring={length/.initial=6, amplitude/.initial=2mm},
	spring/.style={thick,decorate,decoration={aspect=0.5, segment length=\pgfkeysvalueof{/tikz/spring/length}, amplitude=\pgfkeysvalueof{/tikz/spring/amplitude},coil}},
 	%%% MASS STYLING
	mass/.style={ fill=red }
}

\section{Transition from a discrete to continuous system}
\label{sec:trans-from-discr}

Consider an infinite chain of equal mass particles spaced $a$ apart,
and connected by uniform, massless springs of force constant $k$,
illustrated in figure \ref{fig:springs}.

\begin{figure}[b]
  \centering
  \begin{tikzpicture}[scale=0.75]
	\foreach \x in {2,4,...,6}{
		\draw [help lines]  (\x, -2) -- (\x, 1);
		\draw[spring={length=6}] (\x,0) -- (\x+2,0);
		\fill [mass] (\x,0) circle (0.1);
	}   \draw [dotted, spring] (0,0) -- (2,0) (8,0) -- (10,0);
	    \fill [mass] (8,0) circle (0.1);
	    \draw [<->] (2,0.8) -- (4,0.8) node [midway, fill=white] {$a$};

	                                   \draw[dotted, spring={length=2}] (0,-1) -- (2.5,-1);
	\fill [mass] (2.5, -1) circle (0.1); \draw[spring={length=2}] (2.5,-1) -- (3.5,-1);
	\fill [mass] (3.5, -1) circle (0.1); \draw[spring={length=2}] (3.5,-1) -- (6.5,-1);
	\fill [mass] (6.5, -1) circle (0.1); \draw[dotted, spring={length=2}] (6.5,-1) -- (10,-1);

	\draw [help lines] (2.5, -1) -- (2.5, -2); 
	\draw [<->] (2, -2) -- (2.5, -2) node [below, midway] {$\eta_{i-1}$};
	\draw [help lines] (3.5, -1) -- (3.5, -2);
	\draw [<->] (3.5, -2) -- (4, -2) node [below, midway] {$\eta_{i}$};
	\draw [help lines] (6.5, -1) -- (6.5, -2);
	\draw [<->] (6.5, -2) -- (6, -2) node [below, midway] {$\eta_{i+1}$};
\end{tikzpicture}
  \caption{A system of connected masses in a one-dimensional infinite system.}
  \label{fig:springs}
\end{figure}

The kinetic energy in the chain is given by
\begin{subequations}
  \begin{equation}
    \label{eq:61}
    T = \half \sum_i m_i \dot{\eta_i}^2
  \end{equation}
  where $\eta_i$ is the displacement of the $i$th particle from its
  equilibrium position. The potential energy from the springs is
  \begin{equation}
    \label{eq:62}
    V = \half \sum_i k (\eta_{i+1} - \eta_i)^2
  \end{equation}
\end{subequations}
The Lagrangian for the system is then
\begin{align}
  L & = T - V                                                                                          \nonumber\\
    & = \half \sum \qty( m \dot{\eta}^2 - k (\eta_{i+1} + \eta_i)^2 )                                  \nonumber\\
    & = \half \sum_i a \qty[ \frac{m}{a} \dot{\eta}_i^2 - k a \qty( \frac{\eta_{i+1} - \eta_i}{a} )^2] \nonumber\\
    & = \sum_i a L_i
\end{align}
where $a$ is the equilibrium separation of the masses. Thus
\begin{equation}
  \label{eq:63}
  \frac{m}{a} \ddot{\eta}_i - ka \qty( \frac{\eta_{i+1} - \eta_i}{a^2} ) + ka \qty( \frac{\eta_i - \eta_{i-1}}{a^2} ) = 0
\end{equation}
As $a \to 0$ then $\frac{m}{a} \to \mu$, for $\mu$ the linear mass
density. Then, from Hooke's Law
\begin{equation}
  \label{eq:64}
  F = Y \xi = k( \eta_{i+1} - \eta_i ) = k a \xi
\end{equation}
for $Y$ the Young's modulus, and $\xi$ the relative extension,
\[ \xi = \frac{\eta_{i+1} - \eta_i}{a} \] Then, in the continuous
limit $x_i$ becomes a coordinate, so $x_i \to x$, so
\[ \frac{\eta_{i+1} - \eta_i}{a} \to \frac{\eta(x+a) - \eta(x)}{a} \to \dv{\eta}{x} \]
as $a \to 0$. Thus
\begin{equation}
  \label{eq:65}
  L = \half \int \qty[ \mu \dot{\eta}^2 - Y \qty( \dv{\eta}{x} )^2] \dd{x}
\end{equation}
and as $a \to 0$,
\[ 
\lim_{a \to 0} \qty( - \frac{Y}{a} \qty[ \eval{\dv{\eta}{x}}_x - \eval{ \dv{\eta}{x}  }_{x-a} ] ) \to \dv[2]{\eta}{x} Y
\]
Thus
\begin{equation}
  \label{eq:66}
  \mu \dv[2]{\eta}{t} - Y \dv{\eta}{x} = 0 \iff \mathcal{L} = \half \qty[\mu \qty(\dv{\eta}{t})^2 - Y \qty(\dv{\eta}{x})^2]
\end{equation}
Where $\mathcal{L}$ is the Lagrangian density. For an elastic rod, the
$\mathcal{L}$ has an $\dv{\eta}{t}$ and $\dv{\eta}{x}$ term, so
$\vec{x}$ and $t$ can have similar roles. If a local force was present
there would be a dependence on $\eta$ and $\nabla \eta$, so a general
form for $\mathcal{L}$ is
\begin{equation}
  \label{eq:67}
  \mathcal{L} = \mathcal{L}\qty(\eta, \dv{\eta}{x}, \dv{\eta}{t}, x, t) 
\end{equation}

\section{The principle of least action for fields}
\label{sec:princ-least-acti}

The principle of least action is now
\begin{equation}
  \label{eq:68}
  \delta I = \delta \int_1^2 \int \mathcal{L} \dd{x} \dd{t} = 0
\end{equation}

Noting that $x$ and $t$ are unaffected by variations, as they have
fixed end points, then the variation in $\eta$ is
\begin{equation}
  \label{eq:69}
  \eta(x,t; \alpha) = \eta(x, t; 0) + \alpha \zeta(x,t)
\end{equation}
Where $\eta(x,t;0)$ is the correct solution, and $\zeta(x,t)$
disappears at $x_1, x_2, t_1, t_2$. Also, for $I$ to be a function of
$\alpha$, and to be an extremum of $\eta(x,t,;0)$ its derivative with
respect to $\alpha$ must vanish at $\alpha=0$.

\begin{align}
  \label{eq:70}
  \dv{I}{\alpha} = \int_{t_1}^{t_{2}} \int_{x_1}^{x_2} \dd{x} & \dd{t} \bigg[ \pdv{\mathcal{L}}{\eta} \pdv{\eta}{\alpha} + \pdv{\mathcal{L}}{\pdv{\eta}{t}} \pdv{\alpha} \qty(\dv{\eta}{t}) + \nonumber\\
& \qquad \pdv{\mathcal{L}}{\pdv{\eta}{x}} \pdv{\alpha} \qty( \dv{\eta}{x} ) \bigg]
\end{align}
We know $a \zeta$ disappears at the end-points, so
\begin{subequations}
  \begin{align}
    \int_{t_1}^{t_2} \pdv{\mathcal{L}}{\pdv{\eta}{t}} \pdv{\alpha} \qty( \dv{\eta}{t} ) \dd{t} &= - \int_{t_1}^{t_2} \dv{t} \qty( \pdv{\mathcal{L}}{\pdv{\eta}{t}} ) \pdv{\eta}{\alpha} \dd{t} \\
\int_{x_1}^{x_2} \pdv{\mathcal{L}}{\pdv{\eta}{x}} \pdv{\alpha} \qty( \dv{\eta}{x} ) \dd{x} &= - \int_{x_1}^{x_2} \dv{x} \qty( \pdv{\mathcal{L}}{\pdv{\eta}{x}} ) \pdv{\eta}{\alpha} \dd{x} 
  \end{align}
Thus
\[ \iint \qty[ \pdv{\mathcal{L}}{\eta} - \dv{t} \qty( \pdv{\mathcal{L}}{\pdv{\eta}{t}} ) - \dv{x} \qty( \pdv{\mathcal{L}}{\pdv{\eta}{x}} ) ] \eval{\pdv{\eta}{\alpha}}_0 = 0 \]
\end{subequations}
So,
\begin{equation}
  \label{eq:71}
  \dv{t} \pdv{\mathcal{L}}{\dot{\eta}} + \dv{x} \pdv{\mathcal{L}}{\eta'} - \pdv{\mathcal{L}}{\eta} = 0
\end{equation}
are the Euler-lagrange equations for a continuous system. Note that
equation \eqref{eq:71} has the form of a single equation, but since
$\eta$ is a differential equation it provides the many solutions to
make the combination an infinite system. Adopting the comma notation
for field derivatives,
\begin{equation}
  \label{eq:72}
  \mathcal{L} = \mathcal{L}(\eta_{\rho}, \eta_{\rho, \nu} x^{\nu}), \qquad L = \int \mathcal{L} (\dd{x^i})
\end{equation}
and the principle of least action has the form
\[ \delta I = \delta \int \mathcal{L}(\dd{x^{\mu}}) = 0 \]
thus
\begin{equation}
  \label{eq:73}
  \dv{x^{\nu}} \qty[ \pdv{\mathcal{L}}{\eta_{\rho, \nu}}] - \pdv{\mathcal{L}}{\eta_{\rho}} = 0
\end{equation}
are the General Euler-Lagrange equations for the 4-vector field.

\section{The stress-energy tensor}
\label{sec:stress-energy-tensor}

An analogue to the energy function then exists,
\begin{align*}
  \pdv{\mathcal{L}}{x^{\mu}} &= \pdv{\mathcal{L}}{\eta_{\rho}} \eta_{\rho,\mu} + \pdv{\mathcal{L}}{\eta_{\rho, \nu}} \eta_{\rho, \mu\nu}+\pdv{\mathcal{L}}{x^{\mu}} \\
&= \dv{x^{\nu}} \qty[\pdv{\mathcal{L}}{\eta_{\rho,\nu}}] \eta_{\rho,\mu} + \pdv{\mathcal{L}}{\eta_{\rho,\nu}} \dv{\eta_{\rho,\mu}}{x^{\nu}} + \pdv{\mathcal{L}}{x^{\mu}} + \pdv{\mathcal{L}}{x^{\mu}} \\ &\qquad\text{(using the Euler-Lagrange equation)} \\
&= \dv{x^{\nu}} \qty[ \pdv{\mathcal{L}}{\eta_{\rho,\nu}} \eta_{\rho,\mu}] + \pdv{\mathcal{L}}{x^{\mu}} \\
- \pdv{\mathcal{L}}{x^{\mu}} &= \dv{x^{\nu}} \qty[ \pdv{\mathcal{L}}{\eta_{\rho, \nu}} \eta_{\rho,\mu} - \mathcal{L} \delta_{\mu \nu}] \tag{\(\star\)}
\end{align*}
Suppose the field doesn't depend upon $x^{\mu}$; a free-field (with no
driving forces or sinkes at explicit points, so doesn't interact with
particles) then ($\star$) has the form of divergence equations,
\[ \tensor{T}{_{\mu}^{\nu}_{,\nu}} = 0 \]
where
\begin{equation}
  \label{eq:74}
  \tensor{T}{_{\mu}^{\nu}} = \pdv{\mathcal{L}}{\eta_{\rho, \nu}} \eta_{\rho, \mu} - \mathcal{L} \delta_{\mu}^{\nu}
\end{equation}
is the stress-energy tensor.

The components represent a number of important quantities,

\begin{centering}
    \begin{tikzpicture}
        \matrix [matrix of math nodes,left delimiter=(,right delimiter=)] (m)
        {
            T^{00} & T^{01} & T^{02} & T^{03} \\
            T^{10} & T^{11} & T^{12} & T^{13} \\
            T^{20} & T^{21} & T^{22} & T^{23} \\
            T^{30} & T^{31} & T^{32} & T^{33} \\
        };  
        \fill[opacity=0.2, color=accent-blue] 
             (m-1-1.north west) -- (m-1-1.north east) -- (m-1-1.south east) -- (m-1-1.south west) -- (m-1-1.north west);
        \fill[opacity=0.2,color=accent-red] 
             (m-1-2.north west) -- (m-1-4.north east) -- (m-1-4.south east) -- (m-1-2.south west) -- (m-1-2.north west);
        \fill[opacity=0.2,color=accent-red] 
             (m-2-1.north west) -- (m-2-1.north east) -- (m-4-1.south east) -- (m-4-1.south west) -- (m-2-1.north west);
        \fill[opacity=0.2,color=accent-yellow] 
             (m-2-2.north west) -- (m-4-2.south west) -- (m-4-4.south east) -- (m-3-2.north east);
\fill[opacity=0.2, color=accent-purple]
(m-2-4.north east) -- (m-4-4.south east) -- (m-2-2.north west) -- (m-2-4.north east);
        \fill[opacity=0.2, color=accent-green] 
             (m-2-2.north west) -- (m-2-2.south west) -- (m-4-4.south west) -- (m-4-4.south east)
          -- (m-4-4.north east) -- (m-2-2.north east) -- cycle;
      \end{tikzpicture}

These are 
{ \color{accent-blue} energy density },
{ \color{accent-red} momentum density},
{ \color{accent-yellow} momentum flux},
{\color{accent-green} pressure}, and
{\color{accent-purple} shear stress}.
    \end{centering}
    
%%% Local Variables: 
%%% mode: latex
%%% TeX-master: "../project"
%%% End: 


\chapter{The Hamiltonian Formalism $\star$}
\label{cha:hamilt-form-star}
In the Lagrangian formalism the equations of motion are given by
\begin{equation}
  \tag{\ref{eq:21}}
  \dv{t}\qty(\pdv{L}{\dot{q}_i}) - \pdv{L}{q_i} = 0
\end{equation}
These are second-order differential equations, and so $2n$ initial
values are required for a full solution, with an $n$-dimensional
configuration space. The Hamiltonian approach is to recast the
equations of motion as first-order equations, with a configuration
space of $2n$ independent variables, describing the position of a
point is spacetime, and the conjugate momenta. Now $(p, q)$ are the
canonical variables.

\section{The Legendre Transform}
\label{sec:legendre-transform}

In order to switch from the parameters of the Lagrangian formalism,
$(q, \dot{q}, t)$ to those of the Hamiltonian, $(q, p, t)$ we
introduce a transformation.

Consider a function of the form
\[ \dd{f} = u \dd{x} + v \dd{y}, \qquad u= \pdv{f}{x}, \quad
v=\pdv{f}{y} \] We want to change from using $x$ and $y$ in the
description to using $u$ and $y$, so let
\[ g = f - ux \]
which has a differential,
\[ \dd{g} = \dd{f} - u \dd{x} - x \dd{u} = v \dd{y} - x \dd{u} \] Thus
$x$ and $v$ are now functions of $u$ and $y$:
\[ x = - \pdv{g}{u}, \qquad v = \pdv{g}{y} \]

\begin{example}[Legendre transforms in thermodynamics]
  Consider the first law of thermodynamics,
  \[ \dd{U}= \dd{Q} - \dd{W} \] For a gas undergoing a reversible
  process this can be re-expressed as
  \[ \dd{U} = T \dd{S} - P \dd{V} \] For the entropy, $S$, and volume
  $V$. The temperature and the pressure are given 
  \[ T = \pdv{U}{S} \qquad P = - \pdv{U}{V} \] To find the enthalpy,
  $H(S,P)$ we use a Legendre transform,
  \[ H = U + PV \] which gives \[ \dd{H} = T \dd{S} + V \dd{P}\]
  where \[ T = \pdv{H}{S}\ \qquad V = \pdv{H}{P} \]
\end{example}

\section{The Hamiltonian}
\label{sec:hamiltonian}

The Hamiltonian function is generated from the Lagrangian using a
Legendre transform, starting with the differential of $L$,
\begin{equation}
  \label{eq:83}
  \dd{L} = \pdv{L}{q_i} \dd{q_i} + \pdv{L}{\dot{q}_i} \dd{\dot{q}_i} + \pdv{L}{t} \dd{t}
\end{equation}
Recalling that $p_i = \pdv*{L}{q_i}$, then
\begin{equation}
  \label{eq:84}
  \dd{L} = \dot{p}_i \dd{q}_i + p_i \dd{\dot{q}}_i + \pdv{L}{t} \dd{t}
\end{equation}
and we can transform to the Hamiltonian using
\begin{subequations}
\begin{equation}
  \label{eq:85}
  H(q, p, t) = \dot{q}_i p_i - L(q, \dot{q}, t)
\end{equation}
with differential
\begin{equation}
  \label{eq:86}
  \dd{H}= \dot{q}_i \dd{p_i} - \dot{p}_i \dd{q}_i - \pdv{L}{t}
\end{equation}
and so
\begin{equation}
  \label{eq:88}
  \dd{H} - \pdv{H}{q_i} \dd{q_i} + \pdv{H}{p_i} + \pdv{H}{t} \dd{t}
\end{equation}
\end{subequations}
Thus we have $2n +1$ relations,
\begin{subequations}
  \begin{align}
\label{eq:89}
    \dot{q}_i    & = \pdv{H}{p_i} \\
\label{eq:90}
    - \dot{p}_i  & = \pdv{H}{q_i} \\
\label{eq:91}
    - \pdv{L}{t} & = \pdv{H}{t}
  \end{align}
\end{subequations}
with equations (\ref{eq:89} -- \ref{eq:90}) the \emph{canonical
  equations of Hamilton}, which are the $2n$ first-order equations
which replace the $n$ second-order Lagrange equations.

If the forces involved in the Lagrangian are the result of a
conservative potential, and if the equations with generalised
coordinates don't depend explicitly on time then the Hamiltonian is
equal to the total energy.

From the definition of $H$ in equation (\ref{eq:85}), and in the
manner of equation (\ref{eq:94}),
\begin{equation}
  \label{eq:98}
  H = \dot{q}_i p_i - [L_0(q_i, t) + L_1(q_i, t)\dot{q}_k + L_2(q_i, t) \dot{q}_k \dot{q}_m]
\end{equation}
If the equations defining the generalised coordinates do not
explicitly depend on time, $L_2 \dot{q}_k \dot{q}_m = T$, and if the
forces can be derived from a conservative potential, $L_0 = -V$, and thus 
\begin{equation}
  \label{eq:99}
  H = T + V = E
\end{equation}

\section{Constructing the Hamiltonian}
\label{sec:constr-hamilt}

The procedure for constructing the Hamiltonian is
\begin{enumerate}
\item Construct $L$ in a given set of $q_i$,
\item Define the $p_i$
\item Form the Hamiltonian using equation (\ref{eq:85})
\item Invert the conjugate momenta to gain the $\dot{q}_i$s
\item These are used to eliminate all $\dot{q}_i$ from $H$
\end{enumerate}


%%% Local Variables: 
%%% mode: latex
%%% TeX-master: "../project"
%%% End: 


\chapter{A Covariant Formulation of Electromagnetism}
\label{cha:covar-form-electr}

\section{Four-potential for a field}
\label{sec:four-potential-field}

In electromagnetism the action has two components; the action for the
free particle, and then a term describing the interaction of the
particle with the field. In electromagnetism this interaction is
determined by a single parameter, the charge, $e$, of the particle.

The action for a charge in an electromagnetic field has the form
\begin{equation}
  \label{eq:75}
  S = \int_a^b \qty( -mc \dd{s} + \frac{e}{c} A_i \dd{x}^i )
\end{equation}
The three spatial components of the four-vector, $A^i$ form the vector
potential, $\vec{A}$, while the time-component, $A^0 = \phi$ is the
scalar potential. Introducing the velocity, $\vec{v} =
\dv*{\vec{r}}{t}$, and integrating over $t$,
\begin{equation}
  \label{eq:76}
  S = \int_{t_1}^{t_2} \qty( - mc^2 \sqrt{1 - \frac{v^2}{c^2}} + \frac{e}{c} \vec{A} \vdot \vec{v} - e \phi ) \dd{t}
\end{equation}
and we can then determine the Lagrangian,
\begin{equation}
  \label{eq:77}
  L = - mc^2 \sqrt{1 - \frac{v^2}{c^2}} 
      + \frac{e}{c} \vec{A} \vdot \vec{v} 
      - e \phi
\end{equation}
The derivative of the Lagrangian, $\pdv{L}{\vec{v}}$ is the
generalised momentum of the particle, so
\begin{equation}
  \label{eq:78}
  \vec{P} = \frac{m \vec{v}}{\sqrt{1 - \frac{v^2}{c^2}}} + \frac{e}{c} \vec{A} = \vec{p} + \frac{e}{c} \vec{A}
\end{equation}
and the Hamiltonian is then
\begin{equation}
  \label{eq:79}
  H = \vec{v} \vdot \pdv{L}{\vec{v}} - L = \frac{mc^2}{\sqrt{1 - \frac{v^2}{c^2}}} + e \phi
\end{equation}


%%% Local Variables: 
%%% mode: latex
%%% TeX-master: "../project"
%%% End: 


\chapter{The Lagrangian for Electrodynamics}
\label{cha:lagr-electr}

\chapter{Accelerated Charges}
\label{cha:accelerated-charges}

\appendix

\chapter{Review of Electromagnetism}
\label{cha:revi-electr}

\chapter{Review of Special Relativity}
\label{cha:revi-spec-relat}



\appendices

\end{document}



%%% Local Variables: 
%%% mode: latex
%%% TeX-master: t
%%% End: 
