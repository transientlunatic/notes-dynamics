
\section{Inertial Reference Frames}
\label{sec:inert-refer-fram}

An inertial frame is a coordinate system which is moving with a
constant velocity. The existence of inertial frames amounts to
Newton's first law.

Consider a frame $S(x,y,z)$ and another frame $S^\prime(x^{\prime},
y^{\prime}, z^\prime)$ with velocity $v$ with respect to $S$ along the
$+x$-direction.
The two frames are related by the equations
\begin{align}
  x^{\prime} &= x - vt \\
  y^{\prime} &= y \\
  z^\prime &= z \\
  t^{\prime} &= t
\end{align}
These are an example of a Galilean transformation. We could also have
used rotations.

Now, consider a particle with mass $m$. Observers in $S$ and
$S^{\prime}$ see

\begin{tabular}{lcc}
  & $S$ & $s^{\prime}$ \\
Position & $\vec{x}$ & $\vec{x^{\prime}}$ \\ 
Velocity & $\dv{\vec{x}}{t}$ & $\dv{\vec{x^{\prime}}}{t}$ \\
Acceleration & $\dv[2]{\vec{x}}{t}$ & $\dv[2]{\vec{x^\prime}}{t}$
\end{tabular}
But, from above, 

\begin{fequation}[Addition of Velocities]
 \dv{\vec{x}}{t} = 
 \begin{pmatrix}
   \dv{x}{t} \\ \dv{y}{t} \\ \dv{z}{t}
 \end{pmatrix}
=
\begin{pmatrix}
  \dv{x^{\prime}}{t} \\
 \dv{y^{\prime}}{t} \\
 \dv{z^{\prime}}{t}
\end{pmatrix}
\end{fequation}

\section{Rotating Frames}
\label{sec:rotating-frames}

Consider a frame $S_1$ which is rotating with angular velocity,
$\omega$, about its $z$-axis with respect to an inertial frame, $S_0$,
such that the axes coincide when $t=0$. As such there will be an angle
between the sympathetic axes of $\theta = \omega t$ at any given time.

Now consider a vector $\vec{A}_1$ in $S_1$ which becomes 
%\newcommand{mat}[1]{#1}
\[A_0 = \mat{R}(\omega t) \vec{A}_1 \] i.e. $\vec{A}$ as seen in $S_0$
rotates about the $z$-axis, corresponding to the rotation of the $S_1$
axes. 

In general $A_1$ may vary with $t$, i.e. $A_1$ is not fixed in
$S_1$. Its rate of change in $S_1$ is not simply given by $A_1$. This
seems confusing at first but due to the fact that both the vector and
the coordinate axes in $S_1$ are changing with time. For example, a
fixed vector in $S_1$ has $\dot{A}_1=0$ but $\dot{A}_0 \neq 0$.
%%% Local Variables: 
%%% mode: latex
%%% TeX-master: "../project"
%%% End: 
