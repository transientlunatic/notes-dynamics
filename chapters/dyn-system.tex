Consider a set of particles which each has a mass $m_i$, and position
vector $\vec{r}_i$, then each particle can experience two forces:
\begin{itemize}
\item $\vec{f}~{int}$ --- internal forces which act between particles in a system,
\item $\vec{f}~{ext}$ --- external forces which act on the system from
  outside, e.g. a magnetic flux.
\end{itemize}
The overall motion of a system is disjoint from its internal motions,
thanks to Newton's third law, since $F_{ij} = - F_{ji}$, where
$F_{ab}$ is the force exerted on particle $a$ by particle $b$, and the
force $F_{ij} \propto (\vec{r}_i - \vec{r}_j)$, i.e. acts upon the
shortest path between the particles.

\section{The Centre of Mass}
\label{sec:centre-mass}

The total mass of a system of particles is
\begin{equation}
  \label{eq:5}
  M = \sum_{i=1}^N m_i
\end{equation}
from which the total momentum can be found as 
\begin{equation}
  \label{eq:6}
  \vec{P} = \sum_{i=1}^N \vec{p}_i = \sum_{i=1}^N \dot{\vec{r}}_i m_i
\end{equation}
which, defining the centre of mass, 
\[ \vec{R} = \frac{\sum_i m_i \vec{r}_i}{\sum_i m_i}\]
leads to
\begin{equation}
  \label{eq:7}
  \vec{P} = M \dv{t} \vec{R}
\end{equation}
The force on the $i$th particle is then
\begin{equation}
  \label{eq:8}
  F_i = F_i^{\rm (ext)} + \sum_{j \neq i} \vec{F}_{ji} = \dv{t}\vec{P} 
\end{equation}
so
\[ \dv{t}\vec{P} = \sum_i \qty[ \vec{F}_i^{(\rm ext)} + \sum_{j \neq i} \vec{F}_{ji}] \]
We have
\begin{align*} \sum_i \sum_{j \neq i} \vec{F}_{ji} &= \sum_i \sum_{j < i} \vec{F}_{ji} + \sum_i \sum_{j > i} \vec{F}_{ji} \\ &= \sum_i \sum_{j<i} (\vec{F}_{ji} + F_{ij}) = 0
\end{align*}
Let $\vec{F}^{(\rm ext)} = \sum_i \vec{F}_i^{\rm (ext)}$, then
\begin{equation}
  \label{eq:9}
  F^{\rm (ext)} = \dv{t} P = M \dv[2]{t} \vec{R}
\end{equation}
Thus, the overall motion of a system relies only on the external force
acting upon it.

\section{Angular Momentum}
\label{sec:angular-momentum}

We have 
\[ m_i \dv[2]{t} r_i = \vec{F}_i^{\rm (ext)} + \sum_{j \neq i} \vec{F}_{ji} \]
and the total angular momentum of a system as
\begin{equation}
  \label{eq:10}
  \vec{L} = \sum^N_{i=1} \vec{r}_i \cp \vec{p}_i = \sum_{i=1}^N m_i \vec{r}_i \cp \dot{\vec{r}}_i
\end{equation}
Then
\begin{align}
  \dot{\vec{L}}          & = \dv{t} \sum_{i=1}^N m_i \vec{r}_i \cp \dot{\vec{r}} \nonumber                                                        \\
                         & = \sum_{i=1}^N m_i \dv{t} (\vec{r}_i \cp \dot{\vec{r}} )\nonumber                                                      \\
                         & = \sum_{i=1}^N m_i \vec{r}_i \cp \ddot{\vec{r}}_i \nonumber                                                            \\
                         & = \sum_{i=1}^N \vec{r}_i \cp \vec{F}_i^{\rm (ext)} + \sum_{i=1}^N \sum_{j \neq i} \vec{r}_i \cp \vec{F}_{ji} \nonumber \\
                         & = \vec{G}^{\rm (ext)} + \sum_{i=1}^N \sum_{j>i} (\vec{r} \cp \vec{F}_{ji} + \vec{r}_j \cp \vec{F}_{ij} ) \nonumber     \\
                         & = \vec{G}^{\rm (ext)} + \sum_{i=1}^N \sum_{j>i} (\vec{r}_i - \vec{r}_j) \cp \vec{F}_{ji} \nonumber                      \\
\therefore \dot{\vec{L}} & = \vec{G}^{\rm (ext)}
\end{align}
Thus angular momentum is conserved unless there is an applied torque.

\section{Separation of Kinetic Energy}
\label{sec:separ-kinet-energy}

Let the position of a particle be described relative to the centre of mass, i.e.
\[ \vec{r}_i = \vec{R} + \vec{r}'_i \]
Then
\begin{align*}
  \sum_{i=1}^N m_i \vec{r}'_i &= \sum^N_{i=1} m_i \vec{r}_i - \sum_{i=1}^N m_i R \\
&= M \qty[ \frac{\sum m_i \vec{r}_i}{\sum m_i } - \vec{R}] = 0
\end{align*}
the kinetic energy $T$ is then
\begin{align}
  T &= \half \sum m_i \dv{\vec{r}_i}{t}^2 \nonumber\\
&= \half \sum_{i=1}^N m_i \qty[ \dot{R}^2 + 2 \dot{\vec{r}}'_i \vdot \dot{\vec{R}} + (\dot{\vec{r}}'_i)^2 ] \nonumber\\
&= \half \sum m_i \dot{\vec{R}}^2 + \half \sum m_i \dot{\vec{r}'_i}^2 + \sum m_i \dot{\vec{r}_i'} \nonumber\\
&= \half \sum m_i \dot{\vec{R}}^2 + \half \sum m_i (\dot{\vec{r}}'_i)^2
\end{align}
Thus the kinetic energy is the sum of the internal energies and the
kinetic energy of a single particle with the mass of the whole system.

\section{Separation of Angular Momentum}
\label{sec:separ-angul-moment}

The total angular momentum of a system is
\begin{align}
  \vec{L} &= \sum \vec{r}_i \cp \vec{p}_i \nonumber \\ &= \sum m \vec{r}_i \cp \dot{\vec{r}}_i \nonumber\\
&= \sum m_i (\vec{R} + \vec{r}_i') \cp (\dot{\vec{R}} + \dot{\vec{r'}}_i ) \nonumber\\
&= M \vec{R} \cp \dot{\vec{R}} + \qty[ \sum m_i \vec{r}' ] \cp \dot{\vec{R}} \nonumber\\ & \quad+ \vec{R} \cp \qty[ \sum m_i \dot{\vec{r}}'_i ] + \sum m_i \vec{r}'_i \cp \dot{\vec{r}}'_i \nonumber\\
&= M \vec{R} \cp \dot{\vec{R}} + \vec{L}~{int}
\end{align}
Where $\vec{L}~{int} = \sum m_i \vec{r}'_i \cp \dot{\vec{r}}'_i$, so
\begin{align}
  \dot{\vec{L}} &= \sum r_i \cp \vec{F}_i^{\rm (ext)} \nonumber\\
 &= \underbracket{\vec{R} \cp \sum \vec{F}_i^{\rm (ext)}}_{\text{torque on system}} + \sum \underbracket{\vec{r}'_i \cp \vec{F}_i^{\rm (ext)}}_{\text{torque on each particle}}
\end{align}

%%% Local Variables: 
%%% mode: latex
%%% TeX-master: "../project"
%%% End: 
