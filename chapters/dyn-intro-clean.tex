Classical mechanics is a subject which is hundreds of years old, and
despite its age continues to be an active research area.

Methods and structures in classical mechanics underly more modern
theories. When we think of classical mechanics we think of Newton's
Laws:
\begin{enumerate}
\item Objects at constant velocity remain so if no external forces act on the objects.
\item $F = m a$
\item Action = reaction.
\end{enumerate}
These are easy to apply to simple systems, i.e. one or two particles
with one or two forces. It's difficult to apply these laws in
complicated systems, for example many-particle systems, and continuous
systems. The aim of this course is two-fold; first examining
non-familiar examples in Newtonian mechanics, e.g. non-inertial
frames; and develop an alternative formulation of classical mechanics,
the Lagrangian formalism, which is easier to apply to complex
systems.In particular new structures will become clear in this new
approach which do not appear easily in the Newtonian approach.
%%% Local Variables: 
%%% mode: latex
%%% TeX-master: "../project"
%%% End: 
