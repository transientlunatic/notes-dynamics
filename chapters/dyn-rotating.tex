
\section{Inertial Reference Frames}
\label{sec:inert-refer-fram}

An inertial frame is a coordinate system which is moving with a
constant velocity. The existence of inertial frames amounts to
Newton's first law.

Consider a frame $S(x,y,z)$ and another frame $S^\prime(x^{\prime},
y^{\prime}, z^\prime)$ with velocity $v$ with respect to $S$ along the
$+x$-direction.
The two frames are related by the equations
\begin{align}
  x^{\prime} &= x - vt \\
  y^{\prime} &= y \\
  z^\prime &= z \\
  t^{\prime} &= t
\end{align}
These are an example of a Galilean transformation. We could also have
used rotations.

Now, consider a particle with mass $m$. Observers in $S$ and
$S^{\prime}$ see

\begin{table}[H] \centering
\begin{tabular}{lcc} 
             & $S$                  & $s^{\prime}$               \\ \hline
Position     & $\vec{x}$            & $\vec{x}'$         \\ 
Velocity     & $\dv{\vec{x}}{t}$    & $\dv{\vec{x}'}{t}$ \\
Acceleration & $\dv[2]{\vec{x}}{t}$ & $\dv[2]{\vec{x}'}{t}$
\end{tabular}
\end{table}

But, from above, 

\begin{equation}%[Addition of Velocities]
 \dv{\vec{x}}{t} = 
 \begin{pmatrix}
   \dv{x}{t} \\ \dv{y}{t} \\ \dv{z}{t}
 \end{pmatrix}
=
\begin{pmatrix}
  \dv{x^{\prime}}{t} + v \\
 \dv{y^{\prime}}{t} \\
 \dv{z^{\prime}}{t}
\end{pmatrix}
\end{equation}

Thus, for velocities we interpret this to mean that the velocity of a
particle in another frame is equal to its velocity in its own frame
added to the relative velocity of its frame to the observer's.

Acceleration, on the other hand, is the same in both frames, and is a
frame-independent quantity. The general expression of this idea this
implies that the laws of physics are the same in all inertial frames;
the principle of Galilean relativity.

\section{Rotating Frames}
\label{sec:rotating-frames}

Consider a frame $S_1$ which is rotating with angular velocity,
$\omega$, about its $z$-axis with respect to an inertial frame, $S_0$,
such that the axes coincide when $t=0$. As such there will be an angle
between the sympathetic axes of $\theta = \omega t$ at any given time.

Now consider a vector $\vec{A}_1$ in $S_1$ which becomes 
\newcommand{\mat}[1]{\mathsf{#1}}
\[A_0 = \mat{R}(\omega t) \vec{A}_1 \] i.e. $\vec{A}$ as seen in $S_0$
rotates about the $z$-axis, corresponding to the rotation of the $S_1$
axes. The explicit form of $\mat{R}(\theta)$ is then
\[ \mat{R} = 
\begin{bmatrix}
  \cos(\omega t) & - \sin(\omega t) & 0 \\
\sin(\omega t)   & \cos(\omega t)   & 0 \\
0              & 0              & 1
\end{bmatrix}
\]

In general $A_1$ may vary with $t$, i.e. $A_1$ is not fixed in
$S_1$. Its rate of change in $S_1$ is not simply given by $A_1$. This
seems confusing at first but due to the fact that both the vector and
the coordinate axes in $S_1$ are changing with time. For example, a
fixed vector in $S_1$ has $\dot{A}_1=0$ but $\dot{A}_0 \neq 0$.

The rate of change of the axes introduces an additional term to
velocities in $S_0$; letting
\begin{align*} 
\dd{\vec{A}_0} &= \underbracket{\qty[ \dd{\mat{R}}(\omega t) ] \vec{A}_1}_{\text{From axis rotation.}} + \mat{R}(\omega t) \dd{\vec{A}_1} \\
&= \vec{\omega} \cp \vec{A}_0 \dd{t} + \mat{R}(\omega t) \dd{A_1} \\
&= \vec{\omega} \cp \qty[ \mat{R}(\omega t) \vec{A}_1] \dd{t} + \mat{R}(\omega t) \dd{A_1}
\end{align*}
Since
\[ \qty| \dd{\mat{R}} \vec{A}_1| = \omega \dd{t} \qty|
\vec{A}_0(\omega t) | = \qty| \omega \cp \vec{A}_0 | \dd{t} \] Where
the second part follows from the change in $\vec{A}_0$ being
perpendicular to $\vec{A}_0$ for the infinitessimal interval $\dd{t}$.
The quantity $\vec{\omega} \cp \vec{A}_0$ points in the direction of
$\dd{R} \vec{A}_1$, so given that the rotation is about the same axis
as the finite rotation $\mat{R}(\omega t)$,
\[ \omega \times \qty( \mat{R} \vec{A}_1) = \mat{R} \qty[ \vec{\omega}
\cp \vec{A}_1] \] This implies that the order of rotations is
irrelevant, so
\begin{align}
  \dd{\mat{R}} \vec{A}_1 &= \mat{R}(\omega t) \qty[ \vec{\omega} \cp \vec{A}_1] \dd{t} \nonumber \\
\implies \dd{\vec{A}_0} &= \mat{R}(\omega t) \qty[ \dd{\vec{A}_1} + \vec{\omega} \cp \vec{A}_1] \dd{t} \nonumber \\
\label{eq:2}
\dot{\vec{A}}_0 &= \mat{R}( \omega t ) \qty[ \dot{\vec{A}}_1 + \vec{\omega} \cp \vec{A}_1 ]
\end{align}

\section{The Coriolis and Centrifugal Forces}
\label{sec:cori-centr-forc}

Let a particle have positions $\vec{x}_0$ and $\vec{x}_1$ in the
frames $S_0$ and $S_1$ respectively, with $S_1$ rotating relative to
$S_0$ according to the transformation $\mat{R}(\omega t)$, so
\begin{align*}
  \vec{x}_0 &= \mat{R}(\omega t) \vec{x}_1 \\
&= \mat{R}(\omega t) \qty[ \dot{\vec{x}}_1 + \vec{\omega} \cp \vec{x}_1 ] &\omit\hfill \text{ (by eq. \eqref{eq:2})} \\
\implies \vec{v}_0 &= \mat{R}(\omega t) \qty[ \vec{v}_1 + \vec{\omega} \cp \vec{x}_1 ]
\end{align*}
Letting $\vec{A}_0 = \vec{v}_0$, and $\vec{A}_1 = \vec{v}_1 + \vec{\omega} \cp \vec{x}_1$, then
\begin{align*}
  \dot{\vec{v}}_0 &= \mat{R}(\omega t) \qty[ \dot{\vec{v}}_1 + \vec{\omega} \cp \dot{\vec{x}} + \vec{\omega} \cp \qty( \vec{v}_1 + \vec{\omega} \cp \vec{x}_1 ) ] \\ 
&= \mat{R}(\omega t) \qty[ \dot{\vec{v}}_1 + 2 \vec{\omega} \cp \vec{v}_1 + \vec{\omega} \cp ( \vec{\omega} \cp \vec{x}_1 )]
\end{align*}
Since $\vec{a}_i = \dot{\vec{v}}_i$ is the acceleration observed in
the frame $S_i$, then
\[ \vec{a}_1 = \mat{R}(-\omega t) \vec{a}_0 - 2 \vec{\omega} \cp
\vec{v}_1 - \vec{\omega} \cp ( \vec{\omega} \cp \vec{x}_1 ) \] So the
acceleration is not the same in each frame, and since in the
individual frames Newton's laws are valid, we see two additional
forces,
\begin{subequations}
\begin{align}
  \label{eq:3}
  \vec{f}~{cor} &= - 2 m \vec{\omega} \cp \vec{v}_1 \\
\vec{f}~{cen} &= - m \vec{\omega} \cp \qty( \vec{\omega} \cp \vec{x}_1 )
\end{align}
\end{subequations}
Respectively the Coriolis and Centrifugal forces.

\section{Motion at the Earth's Surface}
\label{sec:motion-at-earths}

The coordinate system on the Earth's surface is defined using two
coordinates, latitude, $\lambda$, and longitude, $\beta$. We also have
a rotation vector, $\vec{\Omega}$. A point close to the surface is
specified by three coordinates: altitude, latitude, and longitude, and
its velocity has the vector 
\[ \vec{v} = \qty( v~{long}, v~{lat}, v~{alt}) \]
an object near the Earth's surface will experience a Coriolis force,
\[ \vec{f}~{cor} = - 2m \vec{\Omega} \cp \vec{v} \]
and we have
\[ \vec{\Omega} = \qty( 0, \Omega \cos(\lambda), \Omega \sin(\lambda) ) \]
Thus
\begin{align} 
\vec{f}~{cor} &= -2m  \qty( 0, \Omega \cos(\lambda), \Omega \sin(\lambda) ) \cp \qty( v~{long}, v~{lat}, v~{alt}) \nonumber\\
&= - 2 m \Omega \begin{pmatrix} v~{alt} \cos(\lambda) - v~{lat} \sin(\lambda)\\ v~{long} \sin(\lambda) \\ - v~{long} \cos(\lambda)
\end{pmatrix} \nonumber \\
& \text{ if } v~{alt} = 0 \nonumber \\
&= -2m \Omega
\begin{pmatrix}
  - v~{lat} \sin(\lambda) \\ v~{long} \sin(\lambda) \\ 0 
\end{pmatrix} \nonumber \\
&= - 2m \qty(0 , 0 , \Omega \sin(\lambda) ) \cp \qty( v~{long}, v~{lat}, 0) \nonumber \\
\therefore \vec{f}~{cor} &= - 2 m \vec{\Omega}~{alt} \cp \vec{v}
\end{align}

\section{The Foucalt Pendulum}
\label{sec:foucalt-pendulum}

The plane of a pendulum's motion will rotate over time in a rotating
reference frame. Let $S_1$ be the rotating reference frame of the
Earth, and $S_2$ be a frame rotating with angular velocity $\omega_2
\propto \Omega~{alt}$. The accelerations in the two frames satisfy 
\begin{equation}
  \label{eq:1}
  \vec{a}_1 = \mat{R} \qty[ \vec{a}_2 + 2 \vec{\omega}_2 \cp \vec{v}_2 + \vec{\omega}_2 \cp ( \vec{\omega}_2 \cp \vec{x}_2 )]
\end{equation}
where $\vec{x}_i$, $\vec{v}_i$, and $\vec{a}_i$ are respectively the
position, velocity, and accleration in the $i$th frame. Assuming both
$\vec{\Omega}$ and $\vec{\omega}_2$ are small, then the centrifugal
term can be neglected, and
\[ \vec{a}_1 = \mat{R} \qty[ \vec{a}_2 + 2 \vec{\omega}_2 \cp \vec{v}_2 ] \]
we know
\[ \vec{a}_1 = \vec{g} - 2 \vec{\Omega}~{alt} \cp \vec{v}_2 \] For
$\vec{g}$ the acceleration due to gravity, so, given that the pendulum
undergoes horizontal motion, and we can neglect its vertical motion,
ignoring quadratic terms,
\[ \vec{v}_1 \approx \mat{R} \vec{v}_2 \]
and so
\[ \vec{g} - 2 \mat{R} \qty[ \vec{\Omega}~{alt} \cp \vec{v}_2 ] =
\mat{R} \qty[ \vec{a}_2 + 2 \vec{\omega}_2 \cp \vec{v}_2 ] \]
thus, multiplying both sides by $\mat{R}^{-1}$,
\begin{equation}
  \label{eq:4}
  \vec{a}_2 = \mat{R}^{-1} \vec{g}
\end{equation}
and so the acceleration of the bob is simply the acceleration due to
gravity transformed into a different frame, and in $S_2$ it appears as
if only gravity acts, so this represents a frame which is rotating to
counteract the rotation of the pendulum's plane, thus demonstrating
that the plane is rotating.
%%% Local Variables: 
%%% mode: latex
%%% TeX-master: "../project"
%%% End: 
