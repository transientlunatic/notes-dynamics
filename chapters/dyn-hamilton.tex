In the Lagrangian formalism the equations of motion are given by
\begin{equation}
  \tag{\ref{eq:21}}
  \dv{t}\qty(\pdv{L}{\dot{q}_i}) - \pdv{L}{q_i} = 0
\end{equation}
These are second-order differential equations, and so $2n$ initial
values are required for a full solution, with an $n$-dimensional
configuration space. The Hamiltonian approach is to recast the
equations of motion as first-order equations, with a configuration
space of $2n$ independent variables, describing the position of a
point is spacetime, and the conjugate momenta. Now $(p, q)$ are the
canonical variables.

\section{The Legendre Transform}
\label{sec:legendre-transform}

In order to switch from the parameters of the Lagrangian formalism,
$(q, \dot{q}, t)$ to those of the Hamiltonian, $(q, p, t)$ we
introduce a transformation.

Consider a function of the form
\[ \dd{f} = u \dd{x} + v \dd{y}, \qquad u= \pdv{f}{x}, \quad
v=\pdv{f}{y} \] We want to change from using $x$ and $y$ in the
description to using $u$ and $y$, so let
\[ g = f - ux \]
which has a differential,
\[ \dd{g} = \dd{f} - u \dd{x} - x \dd{u} = v \dd{y} - x \dd{u} \] Thus
$x$ and $v$ are now functions of $u$ and $y$:
\[ x = - \pdv{g}{u}, \qquad v = \pdv{g}{y} \]

\begin{example}[Legendre transforms in thermodynamics]
  Consider the first law of thermodynamics,
  \[ \dd{U}= \dd{Q} - \dd{W} \] For a gas undergoing a reversible
  process this can be re-expressed as
  \[ \dd{U} = T \dd{S} - P \dd{V} \] For the entropy, $S$, and volume
  $V$. The temperature and the pressure are given 
  \[ T = \pdv{U}{S} \qquad P = - \pdv{U}{V} \] To find the enthalpy,
  $H(S,P)$ we use a Legendre transform,
  \[ H = U + PV \] which gives \[ \dd{H} = T \dd{S} + V \dd{P}\]
  where \[ T = \pdv{H}{S}\ \qquad V = \pdv{H}{P} \]
\end{example}

\section{The Hamiltonian}
\label{sec:hamiltonian}

The Hamiltonian function is generated from the Lagrangian using a
Legendre transform, starting with the differential of $L$,
\begin{equation}
  \label{eq:83}
  \dd{L} = \pdv{L}{q_i} \dd{q_i} + \pdv{L}{\dot{q}_i} \dd{\dot{q}_i} + \pdv{L}{t} \dd{t}
\end{equation}
Recalling that $p_i = \pdv*{L}{q_i}$, then
\begin{equation}
  \label{eq:84}
  \dd{L} = \dot{p}_i \dd{q}_i + p_i \dd{\dot{q}}_i + \pdv{L}{t} \dd{t}
\end{equation}
and we can transform to the Hamiltonian using
\begin{subequations}
\begin{equation}
  \label{eq:85}
  H(q, p, t) = \dot{q}_i p_i - L(q, \dot{q}, t)
\end{equation}
with differential
\begin{equation}
  \label{eq:86}
  \dd{H}= \dot{q}_i \dd{p_i} - \dot{p}_i \dd{q}_i - \pdv{L}{t}
\end{equation}
and so
\begin{equation}
  \label{eq:88}
  \dd{H} - \pdv{H}{q_i} \dd{q_i} + \pdv{H}{p_i} + \pdv{H}{t} \dd{t}
\end{equation}
\end{subequations}
Thus we have $2n +1$ relations,
\begin{subequations}
  \begin{align}
\label{eq:89}
    \dot{q}_i    & = \pdv{H}{p_i} \\
\label{eq:90}
    - \dot{p}_i  & = \pdv{H}{q_i} \\
\label{eq:91}
    - \pdv{L}{t} & = \pdv{H}{t}
  \end{align}
\end{subequations}
with equations (\ref{eq:89} -- \ref{eq:90}) the \emph{canonical
  equations of Hamilton}, which are the $2n$ first-order equations
which replace the $n$ second-order Lagrange equations.

If the forces involved in the Lagrangian are the result of a
conservative potential, and if the equations with generalised
coordinates don't depend explicitly on time then the Hamiltonian is
equal to the total energy.

From the definition of $H$ in equation (\ref{eq:85}), and in the
manner of equation (\ref{eq:94}),
\begin{equation}
  \label{eq:98}
  H = \dot{q}_i p_i - [L_0(q_i, t) + L_1(q_i, t)\dot{q}_k + L_2(q_i, t) \dot{q}_k \dot{q}_m]
\end{equation}
If the equations defining the generalised coordinates do not
explicitly depend on time, $L_2 \dot{q}_k \dot{q}_m = T$, and if the
forces can be derived from a conservative potential, $L_0 = -V$, and thus 
\begin{equation}
  \label{eq:99}
  H = T + V = E
\end{equation}

\section{Constructing the Hamiltonian}
\label{sec:constr-hamilt}

The procedure for constructing the Hamiltonian is
\begin{enumerate}
\item Construct $L$ in a given set of $q_i$,
\item Define the $p_i$
\item Form the Hamiltonian using equation (\ref{eq:85})
\item Invert the conjugate momenta to gain the $\dot{q}_i$s
\item These are used to eliminate all $\dot{q}_i$ from $H$
\end{enumerate}


%%% Local Variables: 
%%% mode: latex
%%% TeX-master: "../project"
%%% End: 
